% \pagebreak[4]
% \hspace*{10cm}
% \pagebreak[4]
% \hspace*{1cm}
% \pagebreak[4]

\chapter{Estado del arte y trabajos relacionados}
\ifpdf
    \graphicspath{{Chapter1/Chapter1Figs/PNG/}{Chapter1/Chapter1Figs/PDF/}{Chapter1/Chapter1Figs/}}
\else
    \graphicspath{{Chapter1/Chapter1Figs/EPS/}{Chapter1/Chapter1Figs/}}
\fi

\section{Marco conceptual}

Espacio para introducir los trabajos que son soporte teórico fuerte de la investigación, por lo general son artículos o libros de uso constante que cuya ventana de tiempo puede variar durante varios años, e incluso décadas. También son considerados herramientas que aun son validas para soluciones muy comunes en un área específica.

\section{Trabajos relacionados}

Espacio para colocar los trabajos relacionados, por lo general su ventana de tiempo no varía desde la fecha actual a unos años hacia atrás (3, máximo 4 años), por lo general soportan mas el problema de investigación que las herramientas documentadas para resolver el problema. Como ejemplo podemos ver una referencia bibliográfica aquí \cite{6955337}.

% ------------------------------------------------------------------------


%%% Local Variables: 
%%% mode: latex
%%% TeX-master: "../thesis"
%%% End: 
