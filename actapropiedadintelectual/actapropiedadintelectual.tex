\def\baselinestretch{1}
\chapter{Acta de propiedad intelectual}
\ifpdf
    \graphicspath{{actapropiedadintelectual/actapropiedadintelectualFigs/PNG/}{actapropiedadintelectual/actapropiedadintelectualFigs/PDF/}{actapropiedadintelectual/actapropiedadintelectualFigs/}}
\else
    \graphicspath{{actapropiedadintelectual/actapropiedadintelectualFigs/EPS/}{actapropiedadintelectual/actapropiedadintelectualFigs/}}
\fi

\def\baselinestretch{1.66}

\begin{center}
\textbf{UNIVERSIDAD DEL CAUCA}

\textbf{FACULTAD DE INGENIERÍA ELECTRÓNICA Y TELECOMUNICACIONES}

\textbf{ACTA DE ACUERDO SOBRE LA PROPIEDAD INTELECTUAL DEL TRABAJO DE GRADO}
\end{center}

En atención al acuerdo del Honorable Consejo Superior de la Universidad del Cauca, número 008 del 23 de Febrero de 1999, donde se estipula todo lo concerniente a la producción intelectual en la institución, los abajo firmantes, reunidos el día \_\_\_ del mes de \_\_\_\_\_\_\_\_\_\_ de \_\_\_\_\_\_\_\_\_ en el salón del Consejo de Facultad, acordamos las siguientes condiciones para el desarrollo y posible usufructo del siguiente proyecto.

Materia del acuerdo: Trabajo de grado para optar el título de Ingeniero en \_\_\_\_\_\_\_\_.

Título del Proyecto: \_\_\_\_\_\_\_\_

Objetivo del proyecto: \_\_\_\_\_\_\_\_

Duración del proyecto: \_\_\_\_\_\_\_\_

Cronográma de actividades: Ver capítulo \ref{chap:actividadesYCronograma}.

Término de vinculación de cada partícipe en el mismo: \_\_\_\_\_\_\_\_

Organismo financiador: \_\_\_\_\_\_\_\_, naturaleza y cuantía de sus aportes \_\_\_\_\_\_\_\_, porcentaje de los costos del trabajo \_\_\_\_\_\_\_\_. 

Los participantes del proyecto, el (los) señor(es) estudiante(s) de pregrado \_\_\_\_\_\_\_ \_\_\_\_\_\_\_\_\_ y \_\_\_\_\_\_\_\_\_\_\_\_\_, identificado(s) con la cédula de ciudadanía número \_\_\_\_\_\_\_\_\_\_\_\_\_ y \_\_\_\_\_\_\_\_\_\_\_\_\_\_ respectivamente, a quien(es) en adelante se le(s) llamara "estudiante(s)", el ingeniero en calidad de Director del trabajo de grado, identificado con la cédula de ciudadanía \_\_\_\_\_\_\_\_\_, a quien en adelante se le llamará "docente", y la Universidad del Cauca, representada por el Decano de la FIET, manifiestan que:

1.- La idea original del proyecto es de \_\_\_\_\_\_\_\_\_\_\_\_\_ quien la propuso y presentó al Departamento de \_\_\_\_\_\_\_\_\_\_\_\_\_\_\_\_, que la aceptó como tema para el proyecto de grado en referencia.

2.- La idea mencionada fue acogida por el(los)  estudiante(s) como proyecto para obtener el grado de ingeniero(s) en \_\_\_\_\_\_\_\_\_\_\_\_\_\_\_\_\_\_\_\_\_\_\_\_\_\_\_\_\_\_, quienes la desarrollarán bajo la dirección del docente.

3.- Los derechos intelectuales y morales corresponden al docente y a los estudiantes.

4.- Los derechos patrimoniales corresponden al docente, a los estudiantes y a la Universidad del Cauca por partes iguales y continuarán vigentes, aún después de la desvinculación de alguna de las partes de la Universidad.

5.- Los participantes se comprometen a cumplir con todas las condiciones de tiempo, recursos, infraestructura, dirección, asesoría, establecidas en el anteproyecto, a estudiar, analizar, documentar y hacer acta de cambios aprobados por el Consejo de Facultad, durante el desarrollo del proyecto, los cuales entran a formar parte de las condiciones generales.

6.- Los estudiantes de comprometen a restituir en efectivo y de manera inmediata a la Universidad los aportes recibidos y los pagos hechos por la Institución  a terceros por servicios o equipos, si el comité de Investigaciones declara suspendido el proyecto por incumplimiento del cronograma o de las demás obligaciones contraídas por los estudiantes; y en cualquier caso de suspensión, la obligación de devolver en el estado en que les fueron proporcionados y de manera inmediata, los equipos de laboratorio, de cómputo y demás bienes suministrados por la Universidad para la realización del proyecto.

7.- El docente y los estudiantes se comprometen a dar crédito a la Universidad y de hacer mención del Fondo de Fomento de Investigación, en los informes de avance y de resultados, y en registro de éstos, cuando ha habido financiación de la Universidad o del Fondo.

8.- Cuando por razones de incumplimiento, legalmente comprobadas, de las condiciones de desarrollo planteadas en el anteproyecto y sus modificaciones, alguno de los participantes deba ser excluido del proyecto, los derechos aquí establecidos concluyen para él.  Además se tendrán en cuenta los principios establecidos en el reglamento estudiantil vigente de la Universidad del Cauca en lo concerniente a la cancelación y la pérdida del derecho a continuar estudios. 

9.- El documento del anteproyecto y las actas de modificaciones si las hubiere, forman parte integral de la presente acta.

10.- Los aspectos no contemplados en la presente acta serán definidos en los términos del acuerdo 008 del 23 de febrero de 1999 expedido por el Consejo Superior de la Universidad del Cauca, del cual los participantes del acuerdo aseguran tener pleno conocimiento. 

~\\~\\~\\

\noindent
\_\_\_\_\_\_\_\_\_\_\_\_\_\_\_\_~\\
Director 
~\\~\\~\\


\noindent
\_\_\_\_\_\_\_\_\_\_\_\_\_\_\_\_~\\
Tesista
~\\~\\~\\

\noindent
 \_\_\_\_\_\_\_\_\_\_\_\_\_\_\_\_~\\
Tesista
~\\~\\~\\


\noindent
 \_\_\_\_\_\_\_\_\_\_\_\_\_\_\_\_~\\
Decano Facultad


%%% ----------------------------------------------------------------------

% ------------------------------------------------------------------------

%%% Local Variables: 
%%% mode: latex
%%% TeX-master: "../thesis"
%%% End: 
